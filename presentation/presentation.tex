% Copyright 2004 by Till Tantau <tantau@users.sourceforge.net>.
%
% In principle, this file can be redistributed and/or modified under
% the terms of the GNU Public License, version 2.
%
% However, this file is supposed to be a template to be modified
% for your own needs. For this reason, if you use this file as a
% template and not specifically distribute it as part of a another
% package/program, I grant the extra permission to freely copy and
% modify this file as you see fit and even to delete this copyright
% notice. 

\documentclass{beamer}
\usepackage[utf8x]{inputenc}
\usepackage[T1]{fontenc}
\usepackage{lmodern}
\usepackage[slovene]{babel}
\usepackage{hyperref}
\newcommand{\vect}[1]{\boldsymbol{#1}}


\selectlanguage{slovene}
% There are many different themes available for Beamer. A comprehensive
% list with examples is given here:
% http://deic.uab.es/~iblanes/beamer_gallery/index_by_theme.html
% You can uncomment the themes below if you would like to use a different
% one:
%\usetheme{AnnArbor}
%\usetheme{Antibes}
%\usetheme{Bergen}
%\usetheme{Berkeley}
%\usetheme{Berlin}
%\usetheme{Boadilla}
%\usetheme{boxes}
%\usetheme{CambridgeUS}
%\usetheme{Copenhagen}
%\usetheme{Darmstadt}
%\usetheme{default}
%\usetheme{Frankfurt}
%\usetheme{Goettingen}
%\usetheme{Hannover}
%\usetheme{Ilmenau}
%\usetheme{JuanLesPins}
%\usetheme{Luebeck}
\usetheme{Madrid}
%\usetheme{Malmoe}
%\usetheme{Marburg}
%\usetheme{Montpellier}
%\usetheme{PaloAlto}
%\usetheme{Pittsburgh}
%\usetheme{Rochester}
%\usetheme{Singapore}
%\usetheme{Szeged}
%\usetheme{Warsaw}

\title{Sensors}

% A subtitle is optional and this may be deleted
\subtitle{Computational topology - group project}

\author[]{Nejc Kišek, Žan Klaneček}
% - Give the names in the same order as the appear in the paper.
% - Use the \inst{?} command only if the authors have different
%   affiliation.

\institute[] % (optional, but mostly needed)
{

  Faculty of computer and information science\\
  University of Ljubljana}

% - Use the \inst command only if there are several affiliations.
% - Keep it simple, no one is interested in your street address.

\date{June, 2018}
% - Either use conference name or its abbreviation.
% - Not really informative to the audience, more for people (including
%   yourself) who are reading the slides online

\subject{Computational Topology}
% This is only inserted into the PDF information catalog. Can be left
% out. 



%\AtBeginSubsection[]
%{
%  \begin{frame}<beamer>{Outline}
%    \tableofcontents[currentsection,currentsubsection]
%  \end{frame}
%+}

% Let's get started
\begin{document}

\begin{frame}
  \titlepage
\end{frame}

\begin{frame}{Outline}
  \tableofcontents
  % You might wish to add the option [pausesections]
\end{frame}

% Section and subsections will appear in the presentation overview
% and table of contents.
\section{Problem description}
\subsection{Problem}

\begin{frame}{Problem description}{}
Number of sensors on the sphere of radius 1 (Earth):

\begin{itemize}
	\item {
		each sensor gathers data from the surrounding area in the shape of a circle of radius
		$R$,
	}
	\item {
		each sensor can communicate with other sensors which are at most $r$ away.
	}
\end{itemize}
\end{frame}
\subsection{Goals}
\begin{frame}{Goals}{}

\begin{enumerate}
	\item {Values of $r$ and $R$ are as small as possible.}
	\item {The sensor network is connected.}
	\item {The sensor network covers the whole sphere.}
	\item {Removal of obsolete sensors.}
	\item {Find optimal distribution of 50 sensors on the sphere.}
\end{enumerate}

\end{frame}

\section{Topological solution}

\subsection{Vietoris-Rips complex}

\begin{frame}{Vietoris-Rips complex}{}
\textbf{Connected sensor network} $\longrightarrow$ Vietoris-Rips complex $VR_\delta(S)$ is connected.
  \begin{itemize}
  	\item {sensors: S $\big(S_i = (r_i, \phi_i, \theta_i)\big)$,}
  	\item {sensor connections $\{S_i, S_j\} \subset S; d(S_i, S_j) \leq 2\delta$,}
  	\item {$F \subset S$ is a simplex in $VR_\delta(S)$, if diam $F \leq 2\delta$}.
  \end{itemize}
\end{frame}
\subsection{Čech complex}

\begin{frame}{Čech complex}{}
\textbf{The sensor network covers the whole sphere} $\longrightarrow$ Euler characteristic of Čech complex should be that of a sphere.
\begin{itemize}
	\item {sensors: S $\big(S_i = (r_i, \phi_i, \theta_i)\big)$,}
	\item {$B_\delta(x)$ closed ball with radius $\delta$ around $x$,}
	\item {$Č_\delta = \{\sigma \subset S,\cap_{x\in \sigma}B_\delta(x) \neq \emptyset \}$}.
\end{itemize}
\end{frame}

\section{Results and implementation}

\subsection{Data generator}
\begin{frame}{Data generator}{}
Distribution of points on the sphere so that parameters $r$ and $R$ are as small as possible.

\begin{alertblock}{Electrostatic potential energy}
	\centering $V = \sum_{i\neq j}V_{ij} \propto \sum_{i\neq j}\frac{1}{|\vect{r_i} - \vect{r_j}|}$
\end{alertblock}
\begin{itemize}
	\item {Electrons would distribute themselves evenly around the sphere. }
	\item {Minimization of $V$ with simulated annealing.}
\end{itemize}

\end{frame}

\subsection{Algorithm for MC simulated annealing}
\begin{frame}{Algorithm for MC simulated annealing}
\begin{enumerate}
	\item {Start with random distribution of points on sphere.}
	\item {Set initial temperature of the system $T$.}
	\item {Choose random point, move it according to Gaussian distribution.}
	\item {Calculate difference in energy $\Delta E$.}
	\item {If $\Delta E < 0$, accept the change.} 
	\item {If $\Delta E \geq 0$, accept the change with probability $\exp(\frac{-\Delta E}{T})$}
	\item {If enough changes accepted, decrease the temperature $T$.}
	\item {Repeat process from 3. $\longrightarrow$}
\end{enumerate}

\end{frame}

\section*{Summary}

\begin{frame}{Summary}
 Baumchiquabaumbaum.
\end{frame}





\begin{frame}
  \frametitle<presentation>{Literature}
    
  \begin{thebibliography}{99}

  \setbeamertemplate{bibliography item}[online]
  	 
  \bibitem{A}Vietoris-Rips. \url{https://en.wikipedia.org/wiki/Vietoris_Rips_complex} (5.6.2018).
  \bibitem{A}Čech-complex. \url{https://en.wikipedia.org/wiki/Cech_complex} (5.6.2018).
  \setbeamertemplate{bibliography item}[book]
 	
  \bibitem{B} Lecture notes from prof. dr. Neža Mramor Kosta.
  \end{thebibliography}
\end{frame}

\end{document}


